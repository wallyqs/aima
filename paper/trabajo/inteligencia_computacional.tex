% Created 2010-05-11 火 23:52
\documentclass[journal]{IEEEtran}
\usepackage[utf8]{inputenc}
\usepackage[T1]{fontenc}
\usepackage{verbatim}



\title{Comparison of the Genetic algorithm and Simulated Annealing for solving TSP problems}
\author{Waldemar Quevedo Salinas  \\a01084463@itesm.mx}
\date{11 5月 2010}

\begin{document}

\maketitle


\begin{abstract}
Simulated Annealing and Genetic Algorithms are two very famous
algorithms for solving combinatorial optimization problems. In this paper, 
each algorithm is applied to a specific instance of a TSP problem
to elucidate what are some of the pros and cons from using either approach.
\end{abstract}

\section{Introduction}
\label{sec-1}



Problem solving is hard. It is said that one of the main reasons of
why it is hard is because we do not know where to even
begin. It is a very creative intense process getting from point A to point B and
figuring out how to get there is a very interesting challenge, for we
have to take into consideration all the possible data and information 
about the world as much as possible. We have to review all the information
provided, work out relationships within these data and to find the way
to connect the goal and with the current information about the problem.

According to Dr.Michalewicsz from his remarkable \emph{How to Solve it} book,
a problem gets complicated because of some of the following reasons:

\begin{itemize}
\item There are so many  possible solutions that the \emph{search space} makes
  exhaustive search unfeasible.
\item The \emph{evaluation function} changes from time to time, requiring not a single
  solution but a set of them
\item Finding a single non-optimal solution is already difficult as it is.
\end{itemize}
So when given the task of solving a problem, we have to take into consideration
the \textbf{search space} of the problem, the \textbf{function to evaluate} 
the quality of a solution and more important what is the \textbf{model} of the
problem.

\section{Traveling Salesman Problem}
\label{sec-2}


The TSP \footnote{Other forms of models are the satisfiability problem (SAT) and non-linear
programming (NLP) } is very useful model when learning algorithms for problem solving 
since it involves a very simple concept: a salesman has to visit certain 
number of cities exactly once time and then return home using the 
shortest path possible. 

The search space of a TSP is the all the possible path that there are 
to solve the unconstrained problem and its evaluation function would be 
the total distance traveled by the traveling salesman from start to finish.








\begin{thebibliography}{1}
\bibitem{IEEEhowto:kopka}
 H.~Kopka and P.~W. Daly, \emph{A Guide to \LaTeX}, 3rd~ed.\hskip 1em plus
 0.5em minus 0.4em\relax Harlow, England: Addison-Wesley, 1999.
\end{thebibliography}

   



\end{document}
